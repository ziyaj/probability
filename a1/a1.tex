\documentclass{article}
\usepackage{amssymb}
\usepackage{amsfonts}
\date{February 2018}

\begin{document}

\section{Polymerase chain reaction}

\subsection{PCR}
Let K denote the number of successes in copying DNA, then we have 
\[
K \sim Bin(n, 0.5)
\]

\subsubsection{what is the probability that there are exactly 8 billion copies immediately after 9 rounds?}

In this particular problem, \( n = 9 \), so \( K \sim Bin(9, 0.5)\)\\
We start from 1 billion, and each success doubles the amount of copies. In the end, there are exactly 8 billion copies. So the number of successes is \( k = {\log_2 8} = 3. \) \\
\\
So the probability is \\
\[
Pr(K = 3) = {9 \choose 3} \cdot 0.5^3 \cdot (1 - 0.5)^{(9-3)} = {9 \choose 3} \cdot (1/2)^9 = 21/128 = 0.1640625
\]

\subsubsection{Find a general formula for the probability that there are exactly x billion copies after y rounds.}

The number of successes after y rounds is \( k = {\log_2 x} \). \\
So the probability is
\[
Pr(K = k) = {y \choose k} \cdot 0.5^k \cdot (1 - 0.5)^{y-k} = {y \choose k} \cdot 0.5^y = 
{y \choose {\log_2 x}} \cdot 0.5^y
\]
k needs to satisfy \( 0 \leq k \leq y\), so the generic formula is: \\
\[
Pr = f(x,y) = \left\{
\begin{array}{ll}
      {y \choose {\log_2 x}} \cdot 0.5^y & , x = 2^k, 0\leq k \leq y, k \in \mathbb{N} \\
      0 & , otherwise\\
\end{array} 
\right.
\]

\subsection{Optional}
\subsubsection{What is the probability that there are exactly 2 billion copies immediately after nine rounds?}

Each successful round the amount doubles, and each failing round the amount is cut by half. In the end, there are exactly 2 billion copies. So we have equation\\
\[ 2^k + (1/2)^{(9-k)} = 2^1 \]
Solving it we get \( k = 5 \). So the probability is \\
\[
Pr(K = 5) = {9 \choose 5} \cdot 0.5^5 \cdot (1-0.5)^{9-5} = {9 \choose 5} \cdot 0.5^9 = 63/256 = 0.24609375
\]

\subsubsection{Find a general formula for the probability that there are exactly x copies after y rounds.}
Following the same logic as in 1.2.1, we have formula: \\
\[
2^k \cdot (1/2)^{(y-k)} = 2^{\log_2 x}
\]
Solving it we get \( k = ({\log_2 x} + y)/2 \). So the probability is:\\
\[
Pr(K = k) = {y \choose k} \cdot 0.5^k \cdot (1 - 0.5)^{y-k} = {y \choose k} \cdot 0.5^y = 
{y \choose ({\log_2 x} + y )/2} \cdot 0.5^y
\]
k needs to satisfy \( 0 \leq k \leq y \), which is \( (0 \leq ({\log_2 x} + y )/2 \leq y)\), so the generic formula is:\\
\[
Pr = f(x,y) = \left\{
\begin{array}{ll}
      {y \choose ({\log_2 x} + y )/2} \cdot 0.5^y & x = 2^j, -y \leq j \leq y, j \in \mathbb{N} \\
      0 & otherwise\\
\end{array} 
\right.
\]

\section{Hypothesis testing}
If we randomly choose 4 cups to be red wine, there are in total \( {8 \choose 4 }\) combinations.
To identify at least 3 of the red wines, it needs to be either choose 3 red wines and 1 white wine \( {4 \choose 3} \cdot {4 \choose 1}\), or choose all 4 red wines \( {4 \choose 4 }\).
The probability is:\\
\[
Pr = \frac{{4 \choose 3} \cdot {4 \choose 1} + {4 \choose 4}}{{8 \choose 4}} = \frac{17}{70}
\]

\section{Gift exchange problem}

\subsection{Suppose there are four people at his party (including himself). In how many ways can the boxes
be distributed so that exactly one person is assigned his/her own gift?}
Suppose we have A, B, C, D four people. We randomly choose 1 person to have his/her own gift, so \({4 \choose 1}\). 
Without loss of generality, we look at the case where A gets his/her own gift. Then B, C, D has to work with a set that each of them does not get his/her own gift, and there are only 2 ways to do that:
\begin{center}
\begin{tabular}{ |c|c|c|c| } 
 \hline
 & B & C & D \\ 
 \hline
 (1) & C & D & B \\ 
 (2) & D & B & C \\ 
 \hline
\end{tabular}
\end{center}
So in total there are \( {4 \choose 1} \cdot 2 = 8\) ways.

\subsection{In how many ways can the boxes be distributed so that no one is assigned his/her own gift? Do not use explicit enumeration in this question. Instead, introduce the number of ways, \(N_i\), of having i people getting their own gift. What is \( \sum_{i=0}^{5} N_i \)?}

Let \( N_i \) be having i people getting their own gift. Then \( N_0 \) means nobody gets his/her own gift, and \( N_1 \) means exactly 1 person gets his/her own gift, ..., etc. \\
So \( \sum_{i=0}^{5} N_i \) should include all the permutations of distributing 5 gifts, so \\
\[ \sum_{i=0}^{5} N_i = 5! \]. \\
Let \( f(x) \) denote the case that in a total of x people, each of them not getting his/her own gift. From the previous question, we know that:\\
\[
\begin{array}{ll}
f(0) = 1 ,\texttt{by default} \\
f(1) = 0, \texttt{no chance for 1 person not getting his/her own gift} \\
f(2) = 1, \texttt{having 2 people and each gets the other's} \\
f(3) = 2, \texttt{from the previous question we know there are 2 ways} \\
\end{array}
\]
Consider 4 people and each not getting his/her own gift, we have\\
\[
\begin{array}{ll}
f(4) = 4! - {4 \choose 1} \cdot f(3) - {4 \choose 2} \cdot f(2) - {4 \choose 3} \cdot f(1) - {4 \choose 4} \cdot f(0) \\
f(4) = 24 - 4 \times 2 - 6 \times 1 - 4 \times 0 - 1 \times 1\\
f(4) = 9
\end{array}
\]
Getting no one is assigned his/her own gift is equivalent to getting \( f(5) = N_0 = 5! - \sum_{i=1}^{5} N_i = 5! - N_1 - N_2 - N_3 - N_4 - N_5.\) \\
\[
\begin{array}{ll}
f(5) = 5! - {5 \choose 1} \cdot f(5-1) - {5 \choose 2} \cdot f(5-2) - {5 \choose 3} \cdot f(5-3) - {5 \choose 4} \cdot f(5-4) - {5 \choose 5} \cdot f(5-5) \\
f(5) = 5! - {5 \choose 1} \cdot f(4) - {5 \choose 2} \cdot f(3) - {5 \choose 3} \cdot f(2) - {5 \choose 4} \cdot f(1) - {5 \choose 5} \cdot f(0) \\
f(5) = 120 - 5 \times 9 - 10 \times 2 - 10 \times 1 - 5 \times 0 - 1 \times 1\\
f(5) = 44
\end{array}
\]
Therefore, there are 44 ways to be distributed so that no one is assigned his/her own gift.

\subsection{If the boxes are randomly distributed in a party of 5, what is the probability that no one is assigned
his/her own gift?}

There are in total \( 5! \) ways of permutation, and 44 ways to satisfy the goal, so the probability is:
\[
p = \frac{44}{5!} = \frac{11}{30}
\]

\subsection{What is the probability that Tom ends up receiving Finn's gift, but Finn does not receive Tom's gift?}

There are 5 people in total in the party, so the total number of permutations are 6! \\
Tom gets Finn's (\textbf{1}). In the remaining 5-1=4 gifts, \textbf{3} of them are not from Tom for Finn to choose. The rest 5-1-1=3 gifts have permutations of \textbf{3!}\\
So the probability is:
\[
p = \frac{1 \times 3 \times 3! }{5!} = \frac{3}{20}
\]

\section{Guessing answers on a test}

\subsection{Your strategy is to not answer false for any two consecutive questions. In how many ways can you answer all ten questions on the test?}

Let \( f(n) \) be in a total of n questions, the number of ways to answer without having false for any two consecutive questions. So the question is asking for \( f(10) \). \\
In a total of n question, if we answer TRUE for the first question, the problem is reduced to \( f(n-1) \). If we answer FALSE for the first question, then in order not to having consecutive false's, we have to answer the second question with TRUE, and the rest of the problem is reduced to \( f(n-2) \). So we have 
\[
f(n) = f(n-1) + f(n-2), n \geq 3 
\]
Consider the base case when n = 1: there are 2 ways so \( f(1) = 2 \). \\
Consider the base case when n = 2: in a total of \( 2^2 = 4 \) ways, FALSE, FALSE is the only invalid case, so \( f(2) = 4 - 1 = 3\). \\
\( f(3) = f(2) + f(1) = 5 \) \\ 
\( f(4) = f(3) + f(2) = 8 \) \\ 
\( f(5) = f(4) + f(3) = 13 \) \\ 
\( f(6) = f(5) + f(4) = 21 \) \\ 
\( f(7) = f(6) + f(5) = 34 \) \\ 
\( f(8) = f(7) + f(6) = 55 \) \\ 
\( f(9) = f(8) + f(7) = 89 \) \\ 
\( f(10) = f(9) + f(8) = 144 \) \\
 (Or we can think that \(f(n) = fibonacci(n+2) \) but we still need to compute if there is no table to consult from)
 Therefore, in 144 ways I can answer all ten questions on the test.

\subsection{Given that Nathan is able to correctly guess the answer to the last two problems, what is the probability
that he passes the test?}

First 8 questions has 60 points, so each question is \( 60 / 8 = 7.5 \) points. \\
Nathan answers the last 2 questions correctly, so he gets at least 40 points for sure. \\
In order to pass the test, Nathan needs to get \( \geq (60 - 40) \) points, and answering 3 questions correctly gets him 22.5 points. \\
So Nathan needs to answer \( x \geq 3 \) questions correctly in the first 8 questions. \\
Let X denote the number of questions Nathan correctly answers in the first 8 questions, we have \( X \sim Bin(8, 0.5)\).
So the probability is:
\[
\begin{array}{ll}
Pr(X \geq 3) = 1 - Pr(X \leq 2) = 1 - Pr(X=0) - Pr(X=1) - Pr(X=2) \\
Pr(X \geq 3) = 1 - {8 \choose 0} \cdot 0.5^0 \cdot (1-0.5)^8 - {8 \choose 1} \cdot 0.5^1 \cdot (1-0.5)^{(8-1)} - {8 \choose 2} \cdot 0.5^2 \cdot (1-0.5)^{(8-2)} \\
Pr(X \geq 3) = 1 - 0.5^8 - 8 \times 0.5^8 - 28 \times 0.5^8 \\
Pr(X \geq 3) = \frac{219}{256} = 0.85546875
\end{array}
\]

\subsection{Find the probability that Nathan passes the test.}

We can divide it into 3 cases:
\begin{enumerate}
\item Nathan answers the last 2 questions correctly.\\
In this case, the probability is \( Pr_{case1} = 0.5^2 \times Pr(X \geq 3) = \frac{219}{1024} \)
\item Nathan answers both of last 2 questions wrong.\\
In this case, Nathan has to answer all first 8 questions correctly. Therefore, the probability is \( Pr_{case2} = 0.5^2 \times 0.5^8 = \frac{1}{1024}\).
\item Nathan answers one of the last 2 questions correctly.\\
The probability of answering one of the last 2 questions correctly is \( {2 \choose 1} \cdot 0.5^1 \cdot (1-0.5)^{(2-1)}= \frac{1}{2} \). \\
In this case, Nathan needs \( \geq 40 \) points in the first question, which means he needs to answer \( \geq 6 \) questions correctly in the first 8 questions. Using the random variable defined in 4.2, the probability for having at least 6 questions correctly answered in the first 8 questions is:
\[
\begin{array}{ll}
Pr(X \geq 6) = Pr(X = 6) + Pr(X = 7) + Pr(X = 8) \\
Pr(X \geq 6) = {8 \choose 6} \cdot 0.5^6 \cdot (1-0.5)^{(8-6)} + {8 \choose 7} \cdot 0.5^7 \cdot (1-0.5)^{(8-7)} + {8 \choose 8} \cdot 0.5^8 \cdot (1-0.5)^{(8-8)} \\
Pr(X \geq 6) = 28 \times 0.5^8 + 8 \times 0.5^8 +  1 \times 0.5^8 \\
Pr(X \geq 6) = 37 \times 0.5^8 \\
Pr(X \geq 6) = \frac{37}{256} \\
\end{array}
\]
So the probability for this case is \( Pr_{case3} = \frac{1}{2} \times \frac{37}{256} = \frac{37}{512} \).
\end{enumerate}


So the total probability for Nathan to pass the exam is:
\[
Pr = Pr_{case1} + Pr_{case2} + Pr_{case3} = \frac{219}{1024} + \frac{1}{1024} + \frac{37}{512} = \frac{294}{1024} = \frac{147}{512} = 0.28109375
\]


\end{document}