\documentclass{article}
\usepackage{amssymb}
\usepackage{amsfonts}
\date{February 2018}

\begin{document}

\section{Polymerase chain reaction}

\subsection{PCR}
Let K denote the number of successes in copying DNA, then we have \( K \sim Bin(n, p), p = 0.5\)

\subsubsection{what is the probability that there are exactly 8 billion copies immediately after 9 rounds?}

In this particular problem, \( n = 9 \), so \( K \sim Bin(9, 0.5)\)\\
We start from 1 billion, and each success doubles the amount of copies. In the end, there are exactly 8 billion copies. So the number of successes is \( k = {\log_2 8} = 3. \) \\
\\
So the probability is \\
\[
Pr(K = 3) = {9 \choose 3} \cdot 0.5^3 \cdot (1 - 0.5)^{(9-3)} = {9 \choose 3} \cdot (1/2)^9 = 21/128 = 0.1640625
\]

\subsubsection{Find a general formula for the probability that there are exactly x billion copies after y rounds.}

The number of successes after y rounds is \( k = {\log_2 x} \). \\
So the probability is \( Pr(K = k) = {y \choose k} \cdot 0.5^k \cdot (1 - 0.5)^{y-k} = {y \choose k} \cdot 0.5^y = 
{y \choose {\log_2 x}} \cdot 0.5^y\) \\
\\
k needs to satisfy \( 0 \leq k \leq y\), so the generic formula is: \\
\[
p = f(x,y) = \left\{
\begin{array}{ll}
      {y \choose {\log_2 x}} \cdot 0.5^y & x = 2^k, 0\leq k \leq y, k \in \mathbb{N} \\
      0 & otherwise\\
\end{array} 
\right.
\]

\subsection{Optional}
\subsubsection{What is the probability that there are exactly 2 billion copies immediately after nine rounds?}

Each successful round the amount doubles, and each failing round the amount is cut by half. In the end, there are exactly 2 billion copies. So we have equation\\
\[ 2^k + (1/2)^(9-k) = 2^1 \]
Solving it we get \( k = 5 \). So the probability is \\
\[
Pr(K = 5) = {9 \choose 5} \cdot 0.5^5 \cdot (1-0.5)^{9-5} = {9 \choose 5} \cdot 0.5^9 = 63/256 = 0.24609375
\]

\subsubsection{Find a general formula for the probability that there are exactly x copies after y rounds.}
Following the same logic as in 1.2.1, we have formula: \\
\[
2^k \cdot (1/2)^{(y-k)} = 2^{\log_2 x}
\]
Solving it we get \( k = ({\log_2 x} + y)/2 \). So the probability is:\\
\[
Pr(K = k) = {y \choose k} \cdot 0.5^k \cdot (1 - 0.5)^{y-k} = {y \choose k} \cdot 0.5^y = 
{y \choose ({\log_2 x} + y )/2} \cdot 0.5^y
\]
k needs to satisfy \( 0 \leq k \leq y , (0 \leq ({\log_2 x} + y )/2 \leq y)\), so the generic formula is:\\
\[
p = f(x,y) = \left\{
\begin{array}{ll}
      {y \choose ({\log_2 x} + y )/2} \cdot 0.5^y & x = 2^j, -y \leq j \leq y, j \in \mathbb{N} \\
      0 & otherwise\\
\end{array} 
\right.
\]

\section{Hypothesis testing}
If we randomly choose 4 cups to be red wine, there are in total \( {8 \choose 4 }\) combinations.
To identify at least 3 of the red wines, it needs to be either choose 3 red wines and 1 white wine \( {4 \choose 3} \cdot {4 \choose 1}\), or choose all 4 red wines \( {4 \choose 4 }\).
The probability is:\\
\[
p = \frac{{4 \choose 3} \cdot {4 \choose 1} + {4 \choose 4}}{{8 \choose 4}} = \frac{17}{70}
\]

\section{Gift exchange problem}

\subsection{Suppose there are four people at his party (including himself). In how many ways can the boxes
be distributed so that exactly one person is assigned his/her own gift?}
Suppose we have A, B, C, D four people. We randomly choose 1 person to have his/her own gift, so \({4 \choose 1}\). 
Without missing of generality, we look at the case where A gets his/her own gift. Then B, C, D has to work with a set that each of them does not get his/her own gift, and there are only 2 ways to do that.
\begin{center}
\begin{tabular}{ |c|c|c|c| } 
 \hline
 & B & C & D \\ 
 \hline
 (1) & C & D & B \\ 
 (2) & D & B & C \\ 
 \hline
\end{tabular}
\end{center}
So in total there are \( {4 \choose 1} \cdot 2 = 8\) ways.

\subsection{In how many ways can the boxes be distributed so that no one is assigned his/her own gift? Do not use explicit enumeration in this question. Instead, introduce the number of ways, \(N_i\), of having i people getting their own gift. What is \( \sum_{i=0}^{5} N_i \)?}

Let \( N_i \) be having i people getting their own gift. Then \( N_0 \) means nobody gets his/her own gift, and \( N_1 \) means exactly 1 person gets his/her own gift, ..., etc. \\
So \( \sum_{i=0}^{5} N_i \) should include all the permutations of distributing 5 gifts, so \\
\[ \sum_{i=0}^{5} N_i = 5! \]. \\
Let \( f(x) \) denote the case that in a total of x people, each of them not getting his/her own gift. From the previous question, we know that:\\
\[
\begin{array}{ll}
f(0) = 1 ,\texttt{by default} \\
f(1) = 0, \texttt{no chance for 1 person not getting his/her own gift} \\
f(2) = 1, \texttt{having 2 people and each gets the other's} \\
f(3) = 2, \texttt{from the previous question we know there are 2 ways} \\
\end{array}
\]
Consider 4 people and each not getting his/her own gift, we have\\
\[
\begin{array}{ll}
f(4) = 4! - {4 \choose 1} \cdot f(3) - {4 \choose 2} \cdot f(2) - {4 \choose 3} \cdot f(1) - {4 \choose 4} \cdot f(0) \\
f(4) = 24 - 4 \times 2 - 6 \times 1 - 4 \times 0 - 1 \times 1\\
f(4) = 9
\end{array}
\]
Getting no one is assigned his/her own gift is equivalent to getting \( f(5) = N_0 = 5! - \sum_{i=1}^{5} N_i = 5! - N_1 - N_2 - N_3 - N_4 - N_5.\) \\
\[
\begin{array}{ll}
f(5) = 5! - {5 \choose 1} \cdot f(5-1) - {5 \choose 2} \cdot f(5-2) - {5 \choose 3} \cdot f(5-3) - {5 \choose 4} \cdot f(5-4) - {5 \choose 5} \cdot f(5-5) \\
f(5) = 5! - {5 \choose 1} \cdot f(4) - {5 \choose 2} \cdot f(3) - {5 \choose 3} \cdot f(2) - {5 \choose 4} \cdot f(1) - {5 \choose 5} \cdot f(0) \\
f(5) = 120 - 5 \times 9 - 10 \times 2 - 10 \times 1 - 5 \times 0 - 1 \times 1\\
f(5) = 44
\end{array}
\]
Therefore, there are 44 ways to be distributed so that no one is assigned his/her own gift.

\subsection{If the boxes are randomly distributed in a party of 5, what is the probability that no one is assigned
his/her own gift?}

There are in total \( 5! \) ways of permutation, and 44 ways to satisfy the goal, so the probability is:
\[
p = \frac{44}{5!} = \frac{11}{30}
\]

\subsection{What is the probability that Tom ends up receiving Finn's gift, but Finn does not receive Tom's gift?}

There are 1 + 5 = 6 people in total in the party, so the total number of permutations are 6! \\
Tom gets Finn's (\textbf{1}). In the remaining 5 gifts, \textbf{4} of them are not from Tom for Finn to choose. The rest of permutations are \textbf{4!}\\
So the probability is:
\[
p = \frac{1 \times 4 \times 4! }{6!} = \frac{2}{15}
\]

\section{Guessing answers on a test}


\end{document}