\documentclass{article}
\usepackage{amsmath}
\title{STAT 302: Bonus for HW}
\author{
	Jin, Ziyang\\
	\texttt{\# 34893140}\
}

\begin{document}
	\maketitle

\noindent A federal committe of three people is to be randomly selected from a group consisting of four Liberals, three Conservatives, and two NDPers (sorry Greens and Bloc). Let L denote the number of Liberals on the committee, and let C denote the number of Conservatives on the committee.\\
\\
(c) [BONUS for HW:] Let N denote the number of NDPers on the committee. Find the joint probability mass function of L, C, and N by directly generalizing our definitions. Compute the marginals for each party.\\
\\
\textbf{Solution}\\
\\
The joint probability function is:\\
\\
Table when N = 0
\begin{center}
 \begin{tabular}{|| c c c c c ||} 
 \hline
 Pr(L=l, C= c) & L=0 & L=1 & L=2 & L = 3\\ [0.5ex] 
 \hline\hline
 C=0 & 0 & 0 & 0 & 4/84 \\ 
 \hline
 C=1 & 0 & 0 & 18/84 & 0 \\
 \hline
 C=2 & 0 & 12/84 & 0 & 0 \\
 \hline
 C=3 & 1/84 & 0 & 0 & 0 \\
 \hline
\end{tabular}
\end{center}
Table when N = 1
\begin{center}
 \begin{tabular}{|| c c c c c ||} 
 \hline
 Pr(L=l, C= c) & L=0 & L=1 & L=2 & L = 3\\ [0.5ex] 
 \hline\hline
 C=0 & 0 & 0 & 12/84 & 0 \\ 
 \hline
 C=1 & 0 & 24/84 & 0 & 0 \\
 \hline
 C=2 & 6/84 & 0 & 0 & 0 \\
 \hline
 C=3 & 0 & 0 & 0 & 0 \\
 \hline
\end{tabular}
\end{center}
Table when N = 2
\begin{center}
 \begin{tabular}{|| c c c c c ||} 
 \hline
 Pr(L=l, C= c) & L=0 & L=1 & L=2 & L = 3\\ [0.5ex] 
 \hline\hline
 C=0 & 0 & 4/84 & 0 & 0 \\ 
 \hline
 C=1 & 3/84 & 0 & 0 & 0 \\
 \hline
 C=2 & 0 & 0 & 0 & 0 \\
 \hline
 C=3 & 0 & 0 & 0 & 0 \\
 \hline
\end{tabular}
\end{center}
The marginal probability functions are:\\
\\
\begin{equation} \label{eq1}
\begin{split}
p_C(c) = & \sum_{all\ l, n} p(c, l, n) \\
 = &\ p(c, 0, 0) + p(c,  0, 1) + p(c, 0, 2) \\
& + p(c, 1, 0) + p(c, 1, 1) + p(c, 1, 2) \\
& + p(c,  2, 0) + p(c, 2, 1) + p(c, 2, 2) \\
\end{split}
\end{equation}
\begin{center}
 \begin{tabular}{|| c c c c c ||} 
 \hline
C=c & C=0 & C=1 & C=2 & C = 3\\ [0.5ex] 
 \hline\hline
$p_C(c)$ & 20/84 & 45/84 & 18/84 & 1/84 \\ 
 \hline
\end{tabular}
\end{center}

\begin{equation} \label{eq2}
\begin{split}
p_L(l) = & \sum_{all\ c, n} p(c, l, n) \\
 = &\ p(0, l, 0) + p(0,  l, 1) + p(0, l, 2) \\
& + p(1, l, 0) + p(1, l, 1) + p(1, l, 2) \\
& + p(2, l, 0) + p(2, l, 1) + p(2, l, 2) \\
& + p(3, l, 0) + p(3, l, 1) + p(3, l, 2) \\
\end{split}
\end{equation}
\begin{center}
 \begin{tabular}{|| c c c c c ||} 
 \hline
L=l & L=0 & L=1 & L=2 & L = 3\\ [0.5ex] 
 \hline\hline
$p_L(l)$ & 10/84 & 40/84 & 30/84 & 4/84 \\ 
 \hline
\end{tabular}
\end{center}

\begin{equation} \label{eq3}
\begin{split}
p_N(n) = & \sum_{all\ c, l} p(c, l, n) \\
 = &\ p(0, 0, n) + p(0, 1, n) + p(0, 2, n) + p(0, 3, n) \\
& + p(1, 0, n) + p(1, 1, n) + p(1, 2, n) + p(1, 3, n) \\
& + p(2, 0, n) + p(2, 1, n) + p(2, 2, n) + p(2, 3, n) \\
& + p(3, 0, n) + p(3, 1, n) + p(3, 2, n) + p(3, 3, n) \\
\end{split}
\end{equation}
\begin{center}
 \begin{tabular}{|| c c c c ||} 
 \hline
N=n & N=0 & N=1 & N=2\\ [0.5ex] 
 \hline\hline
$p_N(n)$ & 35/84 & 42/84 & 7/84\\ 
 \hline
\end{tabular}
\end{center}

\end{document}