%Initial setup
\documentclass[11pt]{article}
\title{STAT302 Assignment \#2 Solutions}
\author{Julian Ho}

%Math Packages
\usepackage{amsmath}
\usepackage{amsfonts}
\usepackage{amsthm,mathabx}

%General formatting Packages
\usepackage{fancyhdr} %To put headers in
\usepackage{multirow} %for tables
\usepackage{graphicx,graphpap,rotate,geometry,subfigure} 
\usepackage{enumitem} %Permits more customisation of lists than enumerate
\usepackage{tikz} %for textcolor

%Document Layout
%Text positioning
\marginparwidth 0pt %
\marginparsep 0pt  %distance between marginal notes box and main text
\oddsidemargin  0pt
\evensidemargin  0pt
\topmargin   0pt
\textwidth   6.8in %control width of text on 8.5X11 page
\textheight  9.50in %control height of text on 8.5X11 page
\voffset -0.8in

\newcommand{\pr}{\text{Pr}}
\newcommand{\e}{\mathbb{E}}
\newcommand{\var}{\text{Var}}
\newcommand{\sd}{\text{SD}}

%Header of the pages
\pagestyle{fancy}
\lhead{STAT 302}
\rhead{Assignment 3}
\setlength{\headheight}{14pt} %to make room for the header vertically due do squashed height
\fancyhfoffset[R]{0.2in} %to stretch header to match the length of the textwidth.


%=============================
\begin{document}

\begin{center}
\textbf{WINTER 2017/18 TERM 2  \,\, STAT 302: ASSIGNMENT 3 \\
Due: 2pm on Tuesday March 27, 2018}
\end{center}

\vspace*{2mm}
\begin{itemize}
  \item
  Please remember to \textbf{include a cover sheet} when you submit your assignment. Please hand in your assignment in the STAT 302 assignment box on the ground floor of the Earth Sciences Building (ESB) by the due time.
  \item
  When answering the questions, writing down the final answer will not
  be sufficient to receive full marks. Please show all calculations
  unless otherwise specified.
  Also define any events and notation that you use in your solutions.
\end{itemize}


\begin{enumerate}[label=\textbf{Question \arabic*:},start=1]

%Question 1:
%===================================================
% Adrian's question
\item
We are studying a collection of molluscs living on a large beach in northern BC. The molluscs live near the high-tide line and it is known that their location is uniformly distributed with minimum -1 and maximum +2 metres from this line. Their daily energy intake (in kilocalories) turns out to be 1 plus 25\% of their squared location.
\begin{enumerate}
	\item What is the cumulative distribution function of the daily energy intake?\\

	\item What is the probability density function of the daily energy intake?\\

	\item What is the expected daily energy intake?\\

\end{enumerate}





%Question 2:
%===================================================
% Julian's question
\item Let $X$ and $Y$ be two independent Bernoulli$(0.5)$ random variables and define $U = X + Y$ and $V = X - Y$. 
\begin{enumerate}
	\item Find the joint and marginal probability mass functions for $U$ and $V$. [It is sufficient to construct a table to describe these mass functions.]\\

	\item Are $U$ and $V$ independent? Why or why not?\\

	\item Find the conditional probability mass functions $p_{U|V = v}(u)$ and $p_{V|U=u}(v)$. [Again, you can construct a table to describe these mass functions.]\\

\end{enumerate}





%Question 3:
%===================================================

\item 
This question will provide an intriguing contrast to Question 2. Recall that if we have a continuous random variable $X$ defined by a pdf $f_X(x)$, and we define a new random variable $Y = g(X)$ where $g$ is a bijective (i.e. one-to-one) transformation, then the inverse of $g$ is well-defined everywhere, $g^{-1}$, and the density of $Y$ is given by $$f_Y(y) = f_X(g^{-1}(y))\cdot \left| \frac d{dy} g^{-1}(y)\right|.$$ We can generalize this to the bivariate setting as follows. Suppose $X_1$ and $X_2$ are continuous random variables with joint pdf $f_{X_1,X_2}(x_1,x_2)$ and suppose that both $u = g_1(X_1,X_2)$ and $v = g_2(X_1,X_2)$ are bijective (i.e. one-to-one) transformations with inverses $g_1^{-1}(u,v)$ and $g_2^{-1}(u,v)$. If these inverse functions have continuous partial derivatives and nonzero {\em Jacobian} 
\[
J = \text{det}
\begin{bmatrix}
	\frac {\partial g_1^{-1}}{\partial u}	& \frac {\partial g_1^{-1}}{\partial v} \\
	\frac {\partial g_2^{-1}}{\partial u}	& \frac {\partial g_2^{-1}}{\partial v}
\end{bmatrix}
= \frac {\partial g_1^{-1}}{\partial u} \frac {\partial g_2^{-1}}{\partial v} - \frac {\partial g_2^{-1}}{\partial u} \frac {\partial g_1^{-1}}{\partial v} \neq 0,
\]
then the joint density of $u$ and $v$ is $$f_{u,v}(u,v) = f_{X_1,X_2}\left( g_1^{-1}(u,v), g_2^{-1}(u,v)\right) |J|,$$ where $|J|$ is the absolute value of the Jacobian.

\begin{enumerate}
  \item Let $X_1$ and $X_2$ be independent standard normal random variables. Write down the joint probability density function of $X_1$ and $X_2$. Moreover, compute $\pr(X_1<1,X_2<1)$.\\

  \item Define the transformations $u = g_1(x_1,x_2) = x_1 + x_2$ and $v = g_2(x_1,x_2) = x_1 - x_2$. Find the inverse functions $g_1^{-1}(u,v)$ and $g_2^{-1}(u,v)$ and compute the Jacobian of this bivariate transformation of variables.\\

  \item Write down the joint pdf of $U = X_1 + X_2$ and $V = X_1 - X_2$ and show that this density separates over variables; i.e. show $f_{U,V}(u,v) = a(u)b(v)$ for some real functions $a(u)$ and $b(v)$. Recall from class that this implies that $U$ and $V$ are actually {\em independent}. \\

  \item Let $X_1,\ldots,X_n$ be a random sample from a normal distribution with mean $\mu$ and variance $\sigma^2$. If $n=2$, show that the sample mean, $\widebar{X}$, and the sample variance, $S_X^2$, are independent random variables. [Hint: first write $\widebar{X}$ and $S_X^2$ in terms of $U$ and $V$ as above, remembering that $U$ and $V$ are linear combinations of {\em standard} normal random variables.]
\end{enumerate}




\vspace*{3mm}

%Question 4:
\item Let $W$ be a Gamma random variable with parameters ($\alpha = 2,\ \lambda = 2$). Conditional on the value $W = w$, $X$ is an exponential random variable with rate parameter $w$.
\begin{enumerate}
\item What is the conditional density function for $X$ given $W = w$? Be sure to indicate any restrictions on the values of $x$ and $w$. If $W=2$, what is the probability that $X\leq 2$?\\

\item What is the probability that $W$ is greater than its expected value? Do {\em not} use an online applet to find this probability; calculate by hand.\\

\item Show that the conditional distribution of $W$ given $X=2$ is Gamma distributed with parameters ($\alpha=3,\ \lambda=4$). [Recall that $\Gamma(\alpha) = (\alpha-1)!$, and use the fact that $f_X(2) = 1/8$.]\\
\end{enumerate}



% Question 5:
%===================================================

\item Let $X$ be the amount of time that a student spends walking from the Earth Sciences Building to the Mathematics Building, and let $Y$ be the amount of time that a student spends walking from the Earth Sciences Building to the Pharmaceutical Sciences Building. Suppose that the joint density of $X$ and $Y$ is given by the following function
\[
f_{X,Y}(x,y) = \left\{
\begin{array}{ll}
\frac {kx}y & \mbox{ if } 0 < x < y < 10 \\
0 & \mbox{ otherwise} 
\end{array}
\right.
\]
for some fixed constant $k$. 
\begin{enumerate}
	\item What value of $k$ makes $f_{X,Y}(x,y)$ an honest probability density function?\\

	\item Find the marginal probability density function for $X$. [Don't forget to specify the support of the function!]\\

	\item Find the probability $\pr(Y + X \leq 10)$.


\end{enumerate}

\end{enumerate}
\end{document}