%Initial setup
\documentclass[11pt]{article}
\title{STAT302 Assignment \#2 Solutions}
\author{Ziyang Jin}

%Math Packages
\usepackage{amsmath}
\usepackage{amsfonts}
\usepackage{amsthm,mathabx}

%General formatting Packages
\usepackage{fancyhdr} %To put headers in
\usepackage{multirow} %for tables
\usepackage{graphicx,graphpap,rotate,geometry,subfigure} 
\usepackage{enumitem} %Permits more customisation of lists than enumerate
\usepackage{tikz} %for textcolor

%Document Layout
%Text positioning
\marginparwidth 0pt %
\marginparsep 0pt  %distance between marginal notes box and main text
\oddsidemargin  0pt
\evensidemargin  0pt
\topmargin   0pt
\textwidth   6.8in %control width of text on 8.5X11 page
\textheight  9.50in %control height of text on 8.5X11 page
\voffset -0.8in

\newcommand{\pr}{\text{Pr}}
\newcommand{\e}{\mathbb{E}}
\newcommand{\var}{\text{Var}}
\newcommand{\sd}{\text{SD}}
\newcommand{\cov}{\text{Cov}}

%Header of the pages
\pagestyle{fancy}
\lhead{STAT 302}
\rhead{Assignment 4}
\setlength{\headheight}{14pt} %to make room for the header vertically due do squashed height
\fancyhfoffset[R]{0.2in} %to stretch header to match the length of the textwidth.


%=============================
\begin{document}

\begin{center}
\textbf{WINTER 2017/18 TERM 2  \,\, STAT 302: ASSIGNMENT 4 \\
Due: 2pm on Tuesday April 10, 2018}
\end{center}


\begin{enumerate}[label=\textbf{Question \arabic*:},start=1]

%Question 1:
%===================================================

\item
Let $Y$ denote the number of bacteria per cubic centimeter in a particular liquid and suppose $Y\sim Poisson(\lambda)$. Further, assume that $\lambda$ varies with location and has a Gamma distribution with parameters $\alpha$ and $\beta$.
\begin{enumerate}
	\item What is the expected number of bacteria per cubic centimeter?\\
	
	Since $Y \sim Poisson(\lambda)$, the expected number of bacteria $\e{(Y)} = \lambda$.\\
	Since $\lambda \sim Gamma(\alpha, \beta)$, the expected value $\e{(\lambda)} = \frac{\alpha}{\beta}$
	

	\item What is the standard deviation of the number of bacteria per cubic centimeter?\\

	\item What is the probability that $Y\geq 1$?\\

\end{enumerate}





%Question 2:
%===================================================

\item Let $X_1$ and $X_2$ be given by the joint probability density function:
\[
f(x_1,x_2) =
\begin{cases}
	6(1-x_2),	& 0\leq x_1\leq x_2\leq 1 \\
	0,		& \text{otherwise}.
\end{cases}
\]
\begin{enumerate}
	\item Find $\cov(X_1,X_2)$.\\
	\begin{align*}
	& f_{X_1}(x_1) = \int_{x_1}^{1} 6 (1 - x_2) dx_2 = 6 (x_2 - \frac{1}{2} x_2^2)_{x_1}^1 = 3 (x_1 - 1)^2,\ 0 \leq x_1 \leq 1 \\
	& \e{(X_1)} = \int_0^1 3x_1(x_1 - 1)^2 dx_1 = 3 \int_0^1 x_1^3 - 2x_1^2 + x_1 dx_1 = \frac{1}{4} \\
	& f_{X_2}(x_2) = \int_0^{x_2} 6(1-x_2) dx_1 = 6(x_2 - x_2^2),\ 0 \leq x_2 \leq 1 \\
	& \e{(X_2)} = \int_0^1 6 x_2 (x_2 - x_2^2) dx_2 = 6 (\frac{1}{3} x_2^3 - \frac{1}{4} x_2^4)_0^1 = \frac{1}{2} \\
	& \e{(X_1X_2)} = \int_0^1 \int_0^{x_2} x_1 x_2 6 (1-x_2) dx_1 dx_2 =  6 \int_0^1 [\frac{1}{2} x_1^2 (x_2 - x_2^2)]_0^{x_2} dx_2 = 3 \int_0^1 x_2^3 - x_2^4 dx_2 = \frac{3}{20} \\
	& Cov(X_1, X_2) = \e{(X_1X_2)} - \e{(X_1)}\e{(X_2)} = \frac{3}{20} - \frac{1}{4} \times \frac{1}{2} = \frac{1}{40}
	\end{align*}
	
	\item Are $X_1$ and $X_2$ independent?\\
	
	$X_1$ and $X_2$ are not independent. We can prove it by contradiction. If $X_1$ and $X_2$ are independent, then $\e{(X_1X_2)} = \e{(X_1)} \cdot \e{(X_2)}$. From (a) we know that $\e{(X_1X_2)} = \frac{3}{20}$, and $\e{(X_1)} \cdot \e{(X_2)} = \frac{1}{8}$,  so $\e{(X_1X_2)} \neq \e{(X_1)} \cdot \e{(X_2)}$. Contradiction. Therefore $X_1$ and $X_2$ are not independent.\\

	\item Compute $\var(3X_1 - X_2)$.\\
	\begin{align*}
	& \e(X_1^2) = \int_0^1 3 x_1^2 (x_1 - 1)^2 d x_1 = 3 \int_0^1 x_1^4 - 2x_1^3 + x_1^2 dx_1 =  3 (\frac{1}{5}x_1^5 - \frac{1}{2}x_1^4 + \frac{1}{3} x_1^3)_0^1 =  \frac{1}{10} \\
	& \var(X_1) = \e(X_1^2) - [\e(X_1)]^2  = \frac{1}{10} - (\frac{1}{4})^2 = \frac{3}{80} \\
	& \e(X_2^2) = \int_0^1 6 x_2^2 (x_2 - x_2^2) dx_2 = 6 (\frac{1}{4} x_2^4 - \frac{1}{5} x_2^5)_0^1 = \frac{3}{10} \\
	& \var(X_2) = \e(X_2^2) - [\e(X_2)]^2 =  \frac{3}{10} - (\frac{1}{2})^2 = \frac{1}{20} 
	\end{align*}
	\begin{align*}
	\var(3X_1 - X_2) & = \var(3X_1) + \var(-X_2) + 2 Cov(3X_1, -X_2) \\
	& = 9 \var(X_1) + \var(X_2) - 6Cov(X_1, X-2) \\
	& = 9 \times \frac{3}{80} + \frac{1}{20} - 6 \times \frac{1}{40} \\
	& = \frac{19}{80}
	\end{align*}

\end{enumerate}





%Question 3:
%===================================================

\item 
Recent Statistics Canada data indicate that among Canadians who do not develop diabetes or hypertension in their lifetimes, the average life expectancy for females is 85.5 years while the average life expectancy for males is 81.0 years. Females also outnumber males in Canada by a slim margin of 50.4\% to 49.6\% of the total population. Suppose we randomly select a Canadian with negligible chance of developing diabetes or hypertension (i.e. healthy diet, active, non-smoker, no family history of afflictions).

\begin{enumerate}
  \item Find a bound on the probability that this person will live to be more than 100 years old.\\

  \item Statistics Canada also reports that the standard deviation of life expectancy in individuals who do not develop diabetes or hypertension is about 7 years for women, and about 8 years for men. Using this new information, find a better bound on the probability that our randomly selected individual will live to be more than 100 years old.\\

  \item Now suppose we select 50 men at random who have a negligible chance of developing diabetes or hypertension. What is the chance that their average life expectancy will be less than 80 years? How does this compare to the chance of the same event if we were to select 50 healthy women instead? \\

\end{enumerate}




\end{enumerate}
\end{document}